\documentclass[oneside,final,12pt]{extreport}
\usepackage[utf8x]{inputenc}
\usepackage[russianb]{babel}
\usepackage{vmargin}
\usepackage{amsmath}
\usepackage{amsfonts}
\usepackage{amssymb}
\setpapersize{A4}
\setmarginsrb{2cm}{1.5cm}{1cm}{1.5cm}{0pt}{0mm}{0pt}{13mm}
\usepackage{indentfirst}
\sloppy
\begin{document}

\centerline{\bf\LargeТеормин по математическому анализу, 3 семестр.}
\noindent\( \{ u_k \}_{k = 1}^\infty, u_k \in \mathbb {R}, \sum\limits_{k=1}^\infty u_k \) называется числовым рядом. \\
\( u_k \) - общий член ряда. \\
\( S_n  = \sum\limits_{k=1}^n u_k \) - n-я частичная сумма ряда \\ 
\begin{enumerate}
    \item Числовой ряд \( \sum\limits_{k=1}^\infty u_k\) называется сходящимся, если сходится последовательность его частичных сумм \( \{ S_n \} \), т.е. \( \exists \lim\limits_{n \to \infty} S_n = S \).
    \[ \forall \varepsilon > 0 \quad \exists N: \forall n \geqslant N \quad | S_n - S | < \varepsilon \]
    
    \item Критерий Коши:
    \[ \sum\limits_{k=1}^\infty u_k \rightarrow \, \Leftrightarrow \forall \varepsilon > 0 \quad \exists N: \forall n \geqslant N, \, \forall p \in \mathpp {N} \]
    \[ | S_{n + p} - S_n | = | \sum\limits_{k=n+1}^{n+p}u_k | < \varepsilon \]
    Таким образом, необходимое условие сходимости: \( \sum\limits_{k=1}^{n+p}u_k \rightarrow \, \Rightarrow \lim\limits_{k \to \infty} u_k = 0 \).
    
    \item 1-й признак сравнения:
    \[ 0 \leqslant p_k \leqslant q_k, \, \forall k \geqslant k_0 \geqslant 1 \Rightarrow \]
    \[ \sum\limits_{k=1}^\infty q_k \rightarrow \, \Rightarrow \sum\limits_{k=1}^\infty p_k \rightarrow\]
    \[ \sum\limits_{k=1}^\infty p_k \not\rightarrow \, \Rightarrow \sum\limits_{k=1}^\infty q_k \not\rightarrow\]

    2-й признак сравнения: \\
    \( \sqsupset p_k > 0, \, q_k > 0, \, k \geqslant k_0 \geqslant 1 \) и \( \exists \lim\limits_{k\to\infty}\frac{p_k}{q_k}=L \ne 0 \Rightarrow \) \\ для \( \forall L \) оба ряда сходятся или расходятся одновременно. \\
    
    3-й признак сравнения: \\
    \( \sqsupset p_k > 0, \, q_k >0, \,\forall k \geqslant k_0 \geqslant 1 \) \\
    \( \frac{p_{k+1}}{p_k} \leqslant \frac{q_{k+1}}{q_k} \Rightarrow \) справедливо утверждение первого признака сравнения. \\
    
    Частный признак: \\
\[ \sqsupset p_k = o^*(\frac{1}{k^\alpha}), \, k \rightarrow \, \Rightarrow\]
    \[ \sum\limits_{k = 1}^\infty p_k \limits_{\not\rightarrow \, \alpha \leqslant 1}^{\rightarrow \, \alpha > 1} \]
    
    \item Признак Даламбера: \\
    \[ \sqsupset p_k > 0, \, \forall k \geqslant k_0 \geqslant 1 \]
    \[\exists\lim\limits_{k\to\infty}\frac{p_{k+1}}{p_k} = q  \Rightarrow\]
    \[ \sum\limits_{k=1}^\infty p_k \, \limits_{\not\rightarrow \, q > 1}^{\rightarrow \, q < 1} \]
    
    \item Признак Коши:
    \[ \sqsupset p_k \geqslant 0, \, \forall k \geqslant k_0 \geqslant 1 \]
    \[\exists\lim\limits_{k\to\infty} \sqrt[k]{p_k} = q  \Rightarrow\]
    \[ \sum\limits_{k=1}^\infty p_k \, \limits_{\not\rightarrow \, q > 1}^{\rightarrow \, q < 1} \]
    
    \item Признак Раабе:
    \[ \sqsupset p_k \geqslant 0, \, \forall k \geqslant k_0 \geqslant 1 \]
    \[\exists\lim\limits_{k\to\infty} k(\frac{p_k}{p_{k+1}} - 1) = q  \Rightarrow\]
    \[ \sum\limits_{k=1}^\infty p_k \, \limits_{\not\rightarrow \, q < 1}^{\rightarrow \, q > 1} \]
    
    \item Признак Гаусса:
    \[ \sqsupset p_k >0, \]
    \[ \frac{p_k}{p_{k + 1}} = \lambda + \frac{\mu}{\kappa} + \frac{\theta_k}{\kappa^{1 + \xi_p}}, |\theta_k| < \mu, \xi_p > 0 \Rightarrow \]
    \[ \sum\limits_{k=1}^\infty p_k \, \limits_{\not\rightarrow \, \lambda < 1; \, \lambda = 1, \,\mu \leqslant 1 }^{\rightarrow \, \lambda > 1; \, \lambda = 1, \, \mu > 1} \]
    
    \item Интегральный признак Коши-Маклорена: \\
    \( \sqsupset f(x) \geqslant 0, \) монотонно не возрастает, \\
    \( \forall x \geqslant m, \, m \in \mathbb{N} \Rightarrow \) \\
    \( \int\limits_m^\infty f(x) \, dx \) и \( \sum\limits_{k = m}^\infty f(k) \) сходятся и расходятся одновременно.
    
    \item Ряд \( \sum\limits_{k=1}^\infty \) называется абсолютно сходящимся, если сходится ряд из модулей \( \sum\limits_{k=1}^\infty | u_k | \). \\
    Ряд \( \sum\limits_{k=1}^\infty \) называется условно сходящимся, если он сходится, а ряд из модулей расходится.
    
    \item Теорема Коши: \\
    Если ряд \( \sum\limits_{k=1}^\infty u_k\) сходится абсолютно, то \( \forall \) ряд, полученный из него посредством некоторой перестановки, сходится абсолютно и имеет ту же сумму, что и ряд \( \sum\limits_{k=1}^\infty u_k\).
    
    \item Теорема Римана: \\
    
    
    
\end{enumerate}

\end{document}